\chapter{Related Work}
\index{Related Work@\emph{Related Work}}

This report draws upon many different concepts in computer architecture cited above, but the three related works we find the most relevant are Hawkeye’s original publication by Jain and Lin \cite{hawkeye}, Koo et al’s Access Pattern-Aware Cache Management (APCM) paper \cite{apcm}, and two patents by Marvell and Nvidia that are similar to our idea of precleaning \cite{preclean_cpu,preclean_nvidia_patent}.

\section{Hawkeye}
Hawkeye, as touched on in the background section, uses OPTGen as an online method for approximating the optimal cache replacement policy. OPTGen is always trying to match two sequential accesses to a cache line over the program's execution. The time between these sequential accesses represents how long a line needs to be in the cache before getting a hit. With a 4-way associative cache OPTGen can only have 4 of these lines overlapping before a line must be evicted or bypassed. OPTGen, like Belady’s OPT, always prefers lines that has its next access the earliest, and thus prefers caching lines with the shortest interval between accesses. When OPTGen decides to forego caching a line, this line’s associated feature is trained negatively (cache unfriendly), and likewise OPTGen positively trains (cache friendly) features associated with lines that are successfully cached. When Hawkeye needs to evict a cache line it prefers to evict the lines that OPTGen determines to be cache unfriendly. When all lines are cache friendly, Hawkeye falls back on RRIP \cite{rrip} to decide which line to evict.

\section{APCM}
APCM is a relatively recent paper on GPU caching improvements that our research draws draws inspiration from in a number of ways. APCM provides some key insights into the differences between CPU and GPU caches. APCM detects when lines are likely to exhibit reuse between warps of the same core. The cache management policy then pins or bypasses lines to maximize reuse and avoid thrashing within the L1 cache.

One of the more interesting points in the paper is its measurement of how sensitive certain GPU benchmarks are to cache performance. Furthermore, APCM also introduces the idea of sampling warps for its replacement policy meta-data. Rather than adding complicated sampling mechanisms to all the L1 caches, APCM instead assumes that GPU programs are regular enough such that a single warp's access patterns is representative of all the other warps. Finally, we note that APCM is notable for this research for affirming the idea that more complicated cache replacement policies can be applied at the L1 without significant performance costs. Typically, we would not see anything more complicated than LRU replacement at the L1 level for latency and cost savings. APCM shows that we can benefit from a relatively more complex scheme at the L1 level.

\section{Precleaning Related Patents}
(Precleaning Patents)
